%%%%%%%%%%%%%%%%%%%%%%%%%%%%%%%%%%%%%%%%%
% Beamer Presentation
% LaTeX Template
% Version 1.0 (10/11/12)
%
% This template has been downloaded from:
% http://www.LaTeXTemplates.com
%
% License:
% CC BY-NC-SA 3.0 (http://creativecommons.org/licenses/by-nc-sa/3.0/)
%
%%%%%%%%%%%%%%%%%%%%%%%%%%%%%%%%%%%%%%%%%

%----------------------------------------------------------------------------------------
%	PACKAGES AND THEMES
%----------------------------------------------------------------------------------------

\documentclass[xcolor=dvipsnames]{beamer}

\mode<presentation> {

% The Beamer class comes with a number of default slide themes
% which change the colors and layouts of slides. Below this is a list
% of all the themes, uncomment each in turn to see what they look like.

%\usetheme{default}
%\usetheme{AnnArbor}
%\usetheme{Antibes}
%\usetheme{Bergen}
%\usetheme{Berkeley}
%\usetheme{Berlin}
%\usetheme{Boadilla}
%\usetheme{CambridgeUS}
%\usetheme{Copenhagen}
%\usetheme{Darmstadt}
%\usetheme{Dresden}
%\usetheme{Frankfurt}
%\usetheme{Goettingen}
%\usetheme{Hannover}
%\usetheme{Ilmenau}
%\usetheme{JuanLesPins}
%\usetheme{Luebeck}
\usetheme{Madrid}
%\usetheme{Malmoe}
%\usetheme{Marburg}
%\usetheme{Montpellier}
%\usetheme{PaloAlto}
%\usetheme{Pittsburgh}
%\usetheme{Rochester}
%\usetheme{Singapore}
%\usetheme{Szeged}
%\usetheme{Warsaw}

% As well as themes, the Beamer class has a number of color themes
% for any slide theme. Uncomment each of these in turn to see how it
% changes the colors of your current slide theme.

%\usecolortheme{albatross}
%\usecolortheme{beaver}
%\usecolortheme{beetle}
%\usecolortheme{crane}
\usecolortheme{dolphin}
%\usecolortheme{dove}
%\usecolortheme{fly}
%\usecolortheme{lily}
%\usecolortheme{orchid}
%\usecolortheme{rose}
%\usecolortheme{seagull}
%\usecolortheme{seahorse}
%\usecolortheme{whale}
%\usecolortheme{wolverine}

%\setbeamertemplate{footline} % To remove the footer line in all slides uncomment this line
%\setbeamertemplate{footline}[page number] % To replace the footer line in all slides with a simple slide count uncomment this line

\setbeamertemplate{navigation symbols}{} % To remove the navigation symbols from the bottom of all slides uncomment this line
}



\input inputs/packages.tex
\input inputs/abbreviations.tex
\input inputs/commands.tex

%----------------------------------------------------------------------------------------
%	TITLE PAGE
%----------------------------------------------------------------------------------------

\newcommand\meeting{Thesis defense}

\title[\meeting]{Measurement of the photon energy spectrum in
inclusive radiative \texorpdfstring{$B$}{B} meson decays using the
hadronic-tagging method} % The short title appears at the bottom of every slide, the full title is only on the title page

\author[Henrikas Svidras]{\texorpdfstring{\footnotesize \underline{Henrikas Svidras}}{Henrikas Svidras}} % Your name
%\subject{\text{\small KEK, Tsukuba}}
\institute[DESY] % Your institution as it will appear on the bottom of every slide, may be shorthand to save space
{
\text{\large \meeting}
\vspace{-10pt}
}
\titlegraphic{\includegraphics[width=2cm]{logos/DESY.png}\hspace*{4.75cm}~%
   \includegraphics[width=2cm]{logos/belle2.png}
}

\date{March 13, 2023} % Date, can be changed to a custom date

\begin{document}

\setbeamercolor{background canvas}{bg=}
{\setbeamertemplate{footline}{} 
\begin{frame}
\titlepage % Print the title page as the first slide
\end{frame}
}
\addtocounter{framenumber}{-1}

%----------------------------------------------------------------------------------------
%	INTRO
%----------------------------------------------------------------------------------------
\section{Introduction}
\begin{frame}{Introduction: radiative \safeB meson decays}
\centering\small
{\normalsize What are \textbf{radiative $\bm{B}$} decays?
\begin{columns}
   \column{0.4\textwidth}
   \input{tikz/btosgamma_Q.tex}
   \column{0.4\textwidth}
   \input{tikz/btosgamma_W.tex}
\end{columns}
}

\begin{itemize}
   \item Rare decays: \textbf{forbidden at tree level} in the Standard Model!
   \item \textbf{New particles} can appear \textbf{in the loops}, one of many examples:
\end{itemize}
{\normalsize
\begin{columns}
   \column{0.4\textwidth}
   \begin{tikzpicture}
    \begin{feynman}
    \vertex (i1){b};
    \vertex[right =1.2cm of i1] (a) ;
    \vertex[right=0.85cm of a] (b);
    \vertex[right=0.85cm of b] (c);
    \vertex[right=1.2cm of c] (o1) {s,d};
    \vertex[below=2em of c] (g1);
    \vertex[right=2em of g1] (o2) {$\gamma$};
    
    \diagram* {
    (i1) -- [fermion] (a) -- [scalar, edge label =\(H^{\pm}\)] (c) -- [fermion] (o1),
    (a) -- [fermion,half left, edge label = \(uct\)] (c),
    (b) -- [photon] (o2),
    };
    \end{feynman}
\end{tikzpicture}
\end{columns}
}
\begin{itemize}
   \item In the case of \btosgamma: $\mathcal{B}\sim10^{-4}$.
   \item[\ra] Accessible with smaller datasets than e.g. $b\to s\ell\ell$, where $\mathcal{B}\sim10^{-6}$
\end{itemize}

\end{frame}

\begin{frame}{Introduction: inclusive \safeB meson decays}
   \centering\small
   {\normalsize What are \textbf{inclusive $\bm{B}$} decays?}

   \begin{itemize}
      \item $B$ mesons are the \textbf{lightest mesons involving a $\bm{b}$ quark}
      \item[\ra] decays always involve one or more flavour/generation changes!
      \item[\ra] plethora of final decay states from $u, d, c, s$ quarks
   \end{itemize}

   \vspace{10pt}

   \begin{columns}
      \column{0.5\textwidth}
      \centering
         \textbf{Exclusive}: pick a particular state and study its properties
            \begin{itemize}
               \item $B^0\to{K^0}^*(892)\gamma$
               \item $B^+\to K^+\mu^+\mu^-$
            \end{itemize}
      \column{0.5\textwidth}
      \centering
         \textbf{Inclusive}: study all states originating from the associated quark
         \begin{itemize}
            \item $B\to X_s \gamma$
            \item $B\to X_u \ell \nu$
         \end{itemize}
   \end{columns}
   
   \vspace{10pt}
   Just a couple from thousands of potential examples!

\end{frame}

\begin{frame}{Introduction: hadronic tagging}
\centering\small
{\normalsize What is \textbf{hadronic tagging}?}

\vspace{10pt}

\begin{columns}
   \column{0.33\textwidth}
   \input tikz/tagging.tex

   \column{0.66\textwidth}
   \begin{itemize}
      \item Method for $\boldsymbol{e^+}\boldsymbol{e^-}$ \textbf{colliders only}!
      \item Use the known initial state of the \epem collision
      \item \textit{Tag} the signal side
      \item the four-momentum constraint allows to \textbf{infer the charge, flavour, four-momentum} of the signal side
   \end{itemize}
\end{columns}


\end{frame}

%----------------------------------------------------------------------------------------
%	THEORY
%----------------------------------------------------------------------------------------
\section{\safeBtoXsdgamma theory}

\begin{frame}{Description of the inclusive decays}

   The decay rate of \BtoXsdgamma involves both weak and strong interactions.
The typical energy scale of these interactions is much lower than the electroweak scale: $\sim\order(m_W)$.
This motivates approximating the interactions mediated by heavy $Z$ and $\Wpm$ bosons by an effective point-like vertex.

   \begin{equation}\label{eq:effective_lagrangian}\nonumber
      \mathcal{L}_{\mathrm{eff}} = \frac{4G_F}{\sqrt{2}}V_{tq}^*V_{tb}\left[\sum^{8}_{i=1}\mathcal{C}_i(\mu)\mathcal{O}_i(\mu)
                                                  + \frac{V^*_{uq}V_{ub}}{V^*_{tq}V_{tb}}\sum^{2}_{i=1}\mathcal{C}_i(\mu)(\mathcal{O}_i(\mu)-\mathcal{O}_i^u(\mu))\right]
  \end{equation}

  \begin{equation*}
    \mathcal{L}_{\mathrm{eff}} \propto
\mathcal{C}_7 \times
\Biggl[
\raisebox{5pt}{
\resizebox{0.085\textwidth}{!}{
\feynmandiagram [small, inline=(b.base), vertical=b to d] {
    a --  b [dot] -- c,
    b -- [boson] d [particle={\LARGE$\gamma$}],
    };
}
}\Biggr]
+
\mathcal{C}_8 \times
\Biggl[
\raisebox{5pt}{
\resizebox{0.085\textwidth}{!}{
\feynmandiagram [small, inline=(b.base), vertical=b to d] {
    a --  b [dot] --  c,
    b -- [boson] d [particle={\LARGE$g$}],
    };
}
}\Biggr]
+
\sum_i^{1,...,6}
\mathcal{C}_i\times
\Biggl[
\raisebox{3pt}{
\resizebox{0.085\textwidth}{!}{
\feynmandiagram [small, inline=(b.base), horizontal=a to c] {
    a --  b [dot] --  c,
    d --  b -- e,
    };
}
}\Biggr]
+\mathrm{corrections}
\end{equation*}

\end{frame}

\begin{frame}{\safeBtoXsdgamma total decay rate calculations}
   
\end{frame}

\begin{frame}{\safeBtoXsdgamma spectrum calculations}
   
\end{frame}

\end{document}
